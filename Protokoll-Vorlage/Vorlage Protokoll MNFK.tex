\documentclass [titlepage] {article}
\usepackage[T1]{fontenc}
\usepackage{eurosym}
\usepackage[utf8]{inputenc}
\usepackage {lmodern}
\usepackage[ngerman]{babel}
\usepackage{lipsum}
\usepackage{titlesec}
\usepackage{geometry}
\usepackage{enumerate}
\usepackage{colortbl}
\usepackage{subfiles}
\usepackage{ulem}

\geometry {a4paper, top= 20mm, bottom=20mm, left=20mm, right=20mm}
\titleformat{\part}{\rmfamily \bfseries \centering \fontsize{17pt}{22pt}\selectfont}{}{0pt}{}
\titleformat{\section}{\fontsize{12pt}{13pt}\bfseries}{TOP \thesection :  }{0pt}{}

\newcolumntype{L}[1]{>{\raggedright\arraybackslash}p{#1}} % linksbündig mit Breitenangabe
\newcolumntype{C}[1]{>{\centering\arraybackslash}p{#1}} % zentriert mit Breitenangabe
\newcolumntype{R}[1]{>{\raggedleft\arraybackslash}p{#1}} % rechtsbündig mit Breitenangabe

\DeclareUnicodeCharacter{20AC}{\euro}

\begin{document}
\newcommand{\vote}[3]{
	\hspace{1ex} \textit{\normalsize{Abstimmung:}} \\[1mm] 
	\begin{tabular}{c|c|c}
		\scriptsize Ja& \scriptsize Nein& \scriptsize Enthaltung \\
		\rule{0pt}{12pt}
		\Large  #1& \Large #2 &\Large #3
	\end{tabular}
	\normalsize}
\newcommand{\verfahren}[1]{
	\textit{#1} \vspace{2mm} }

\part*{Protokoll der Math.-Nat. FK am XX.XX.XXXX}
\section{Begrüßung}
\verfahren{\dots eröffnet die Sitzung um XX:XX Uhr.}\\
Folgende Fachschaften sind anwesend:
\begin{itemize}
	\setlength\itemsep{0,5mm}
	\item Biologie
	\item Chemie
	\item Geographie
	\item Geowissenschaften
	\item Informatik
	\item Lehramt			
	\item Mathematik
	\item Meteorologie
	\item Molekulare Biomedizin
	\item Pharmazie
	\item Physik / Astronomie
\end{itemize}

\section{Festlegung eines Protokollanten}
\verfahren{.... übernimmt die Protokollführung.}

\section{Endgültige Festlegung der Tagesordnung}
\verfahren{Die in der Einladung verschickte Tagesordnung wird einstimmig angenommen.}\\
Endgültige Tagesordnung:

\begin{enumerate}[{TOP} 1{:}]
	\setlength\itemsep{0,1 mm}
	\item Eröffnung
	\item Festlegung eines Protokollanten
	\item Endgültige Festlegung der Tagesordnung
	\item Verabschiedung von Protokollen
	\item Wahlen
	\item Sonstiges
\end{enumerate}

\section{Verabschiedung von Protokollen}
\verfahren{Das Protokoll der Sitzung vom XX.XX.XXXX wird einstimmig angenommen.}

\section{Wahlen}
\begin{itemize}
	\item QV-Kommission der Fakultät (2 Stellvertreter, vertagt)\\
	\verfahren{Es gibt keine Kandidaten, die Wahl wird vertagt.}
	\item Stellv. Vorsitz der MNFK (vertagt)\\[2mm]
	\verfahren{Es gibt keine Kandidaten, die Wahl wird vertagt.}
\end{itemize}
\section{Berichte aus den Gremien}
\begin{itemize}
	\item \textbf{Fakultätsrat}: 	\verfahren{Es gibt nichts zu berichten.}

	\item \textbf{Studienbeirat}: 	\verfahren{Es gibt nichts zu berichten.}

	\item \textbf{QV-Kommission (Fakultät)}:	\verfahren{Es gibt nichts zu berichten.}

	\item \textbf{Finanzkommission}:	\verfahren{Es gibt nichts zu berichten.}

	\item \textbf{Strukturkommission}:	\verfahren{Es gibt nichts zu berichten.}

	\item  \textbf{QV-Kommission (Senat)}:	\verfahren{Es gibt nichts zu berichten.}
\end{itemize}

\section{Sonstiges}
\verfahren{\dots schließt die Sitzung um XX:XX Uhr.}

\end{document}
