\documentclass{article}
\usepackage[T1]{fontenc}                
\usepackage[utf8]{inputenc} 
\usepackage[ngerman]{babel} 
\usepackage{enumerate}
\usepackage{geometry}
\usepackage{titlesec}

\geometry {a4paper, top= 20mm, bottom=20mm, left=20mm, right=20mm}

\titleformat{\part}{\rmfamily \bfseries \centering \fontsize{16pt}{22pt}\selectfont}{}{0pt}{}
\titleformat{\section}{\fontsize{11pt}{13pt}\bfseries}{\S \ \thesection \ }{0pt}{\normalsize}
\titlespacing*{\section}{0pt}{15pt}{4pt}
\begin{document}
	\newcommand{\mnfk}{Mat.-Nat.-FK\ }
	\part*{Geschäftsordnung der Konferenz der mathematisch-naturwissenschaftlichen Fachschaften \\ (MNFK-GO)}
	\fontsize{10pt}{10pt}\selectfont
	\vspace{0,3cm} Diese Geschäftsordnung regelt die Arbeit der Konferenz der mathematisch-naturwissenschaftlichen Fachschaften. Sie ist nicht zu verwechseln mit der Geschäftsordnung der Fachschaftenkonferenz (FKGO), welche die Vollversammlung aller Fachschaften an der RFWU Bonn betrifft.\\
	Soweit sich Formulierungen in dieser Geschäftordnung auf Personen und Personengruppen beziehen, gelten diese unabhängig des verwendeten Genus für alle Menschen.
	\fontsize{11pt}{13pt}\selectfont
	\section{Allgemeines}
	\begin{enumerate}[(1)]
		\setlength\itemsep{0pt}
		\item Die mathematisch-naturwissenschaftliche Fachschaftenkonferenz (Mat.-Nat.-FK) ist die Vollversammlung all jener Fachschaften, die Studierende in Studiengängen vertreten, die an der  \linebreak[4]Mathematisch-Naturwissenschaftlichen Fakultät der Rheinischen Friedrich-Wilhelms Universität Bonn angesiedelt sind.
		\item Insbesondere sind dies die Fachschaften Mathematik, Informatik, Physik/Astronomie, Chemie, Geographie, Steinmann, Meteorologie, Biologie, Pharmazie, Molekulare Biomedizin und Lehramt.
		Maßgeblich für die Zugehörigkeit ist die Fachschaftenliste (Anlage zur FKGO) in ihrer aktuellen Fassung.
		\item Die \mnfk dient der Vernetzung und dem Erfahrungsaustausch der \linebreak mathematisch-naturwissenschaftlichen Fachschaften.
		\item Die \mnfk übernimmt die Vertretung der Fachschaften gegenüber der Fakultät, der restlichen Studierendenschaft und der Öffentlichkeit, sobald es um Belange geht, die die Gesamtheit der Fachschaften betreffen und eine Koordination nötig machen.
		\item Jede Fachschaft ist angehalten, eine Vertretung auf die \mnfk zu entsenden und die Sitzungsprotokolle zu lesen.
	\end{enumerate}
	\section{Vorsitz}
	\begin{enumerate}[(1)]
		\setlength\itemsep{0pt}
		\item Die \mnfk wählt aus ihrer Mitte für ein Jahr einen Vorsitz sowie einen stellvertretenden Vorsitz.
		\item Der stellvertretende Vorsitz kann alle Aufgaben des Vorsitzes übernehmen.
		\item Der Vorsitz nimmt nach Ende seiner Amtszeit seine Aufgaben kommissarisch weiter wahr, bis ein neuer Vorsitz gewählt ist.
		Eine Sitzung zur Neuwahl ist in diesem Fall zum frühestmöglichen Zeitpunkt einzuberufen.
	\end{enumerate}
	\section{Sitzungen}
	\begin{enumerate}[(1)]
		\setlength\itemsep{0pt}
		\item Die \mnfk tagt mindestens zweimal im Semester.
		\item Die Fachschaften und stundentischen Vertretungen werden zu den Sitzungen mindestens eine \linebreak[3]Woche vorher in Textform eingeladen.
		\item Der Vorsitz leitet die Sitzungen.
		\item Jede Fachschaft nimmt mit einer Stimme an den Abstimmungen teil, stimmberechtigt sind die von den Fachschaften entsendeten Vertretungen.
		\item Solange keine anderen Vorgaben in dieser Geschäftordnung oder übergeordneten Satzungen bestehen, erfolgen Beschlüsse in einer Lesung.
		\item Es gelten die weiteren Regelungen der FKGO entsprechend.
	\end{enumerate}
\pagebreak
	\section{Nominierung von Gremienvertretungen}
	\begin{enumerate}[(1)]
		\setlength\itemsep{0pt}
		\item Die \mnfk nominiert die studentischen Vertretungen für den Studienbeirat, die Qualitätsverbesserungskommission, die Finanzkommission, und die Strukturkommission der Fakulät, die studentische Vertretung der math.-nat. Fakulät in der zentralen Qualitätsverbesserungs-\linebreak kommission, sowie die studentischen Vertretungen in weiteren Kommissionen auf Fakultäts- oder Universitätsebene.
		\item Die Fachschaften haben das Vorschlagsrecht. Solange die Größe der Kommissionen es zulässt, soll jede Fachschaft bzw. jede Fachgruppe vertreten sein. Erheben mehrere Fachschaften Anspruch auf Besetzung eines Sitzes und kann keine Einigung erziehlt werden, entscheidet der Vorsitz durch das Los.
		\item Der Vorsitz leitet die Nominierungen an die entsprechenden Stellen weiter und sichert die Benachrichtigung der gewählten Vertretungen.
		\item Die studentischen Vertretungen können jederzeit zurücktreten. Dies ist dem Vorsitz und der entsendenden Fachschaft umgehend in Textform mitzuteilen. Die frei gewordene Position ist auf der nächsten Sitzung der \mnfk neu zu besetzen.
	\end{enumerate}
	\section{Berichte aus den Gremien}
	\begin{enumerate}[(1)]
		\setlength\itemsep{0pt}
		\item Die von der \mnfk entsendeten studentischen Vertretungen berichten soweit möglich regelmäßig von ihrer Arbeit in den Kommissionen und legen grundlegende Entscheidungen der \mnfk vor.
		\item Die studentischen Mitglieder im Fakultätsrat sind angehalten an den Sitzungen der \mnfk teilzunehmen und über ihre Arbeit zu berichten.
	\end{enumerate}
	\section{Weitere Aufgaben}
	\begin{enumerate}[(1)]
		\setlength\itemsep{0pt}
		\item Die \mnfk kann gemäß § 29 Abs. 2 FKGO die Beantragung von Geldern für fachschaftsübergreifende Maßnahmen und Anschaffungen koordinieren. Die Anträge werden gemäß FKGO der Gesamt-Fachschaftschaftenkonferenz vorgestellt und durch diese abgestimmt.
		\item Die \mnfk kann die Schaffung von gemeinsamen Fachschaftenlisten für die Wahlen der Gremien von Universität und Fakultät koordinieren.
	\end{enumerate}
	\section{Schlussbestimmungen}
	\begin{enumerate}[(1)]
		\setlength\itemsep{0pt}
		\item Diese Geschäftsordnung tritt mit ihrer Veröffentlichung in der AKUT in Kraft.
		\item Änderungen der Geschäftsordnung bedürfen einer Zweidrittelmehrheit einer Mat.-Nat.-FK, auf der mindestens 5 der unter §1 Abs. 2 genannten Fachschaften vertreten sind.\\
		Änderungen der Geschäftsordnung sind auf zwei aufeinanderfolgenden Sitzungen der \mnfk zu lesen.
		Vor Beschluss der Änderungen sind die Fachschaften anzuhören und das Fachschaftenkollektiv zu benachrichtigen.
		\item Im Falle einer unplanmäßigen Regelungslücke ist die FKGO entsprechend anzuwenden.
	\end{enumerate}
\end{document}